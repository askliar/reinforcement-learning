\documentclass{exam}
%--------------------------------------
%packages
\usepackage[margin=1.35in]{geometry}
\usepackage{amsmath}
\usepackage[shortlabels]{enumitem} %for: being abled to use {enumerate}[a)]
\usepackage{amsthm}%proof environment
\usepackage{color}
\printanswers
\newcommand{\matr}[1]{\mathbf{#1}}
\usepackage{amsfonts}
\usepackage{hyperref}
\usepackage{amssymb}
\usepackage{todonotes}
\usepackage{dsfont}
\usepackage{url}
\usepackage{graphicx}
\usepackage{epstopdf}
\usepackage{algorithm, algpseudocode}
\usepackage{cleveref}
\usepackage[compact,explicit]{titlesec}
\usepackage{caption}
\usepackage[export]{adjustbox}

\newcommand{\lp}{\left(}
\newcommand{\rp}{\right)}
\newcommand{\lb}{\left[}
\newcommand{\rb}{\right]}

\newcommand{\norm}[1]{\|#1\|}
\newcommand{\likelihood}{\mathcal{L}}
\usepackage{cancel}
\newcommand{\bishop}[1]{Bishop #1}
\newcommand{\newword}[1]{{\bf #1}}
\newcommand{\Data}{\mathcal{D}}
\newcommand{\Model}{\mathcal{M}}
\newcommand{\DataTrain}{\mathcal{D}{{\text train}}}
\newcommand{\DataTest}{\mathcal{D}{{\text test}}}
\newcommand{\N}{N}
\newcommand{\Ntrain}{\N_{train}}
\newcommand{\Ntest}{\N_{test}}
\newcommand{\DataSize}{\N}
\newcommand{\DataIndex}{n}
\newcommand{\cind}{t}
\newcommand{\pind}{\star}

\newcommand{\eye}{{\bf I}}

\newcommand{\Dim}{D}
\newcommand{\DimIndex}{d}

\newcommand{\DimOut}{K}
\newcommand{\DimOutIndex}{k}
\newcommand{\Identity}{\mathds{1}}
\newcommand{\wavep}{\widetilde{p}}
\newcommand{\waveq}{\widetilde{q}}
\newcommand{\intphi}{\int_{\phivec}}
\newcommand{\inttheta}{\int_{\thetavec}}
\DeclareMathOperator*{\E}{\mathbb{E}}
\DeclareMathOperator*{\ED}{\mathbb{E}_{\Data}}
\DeclareMathOperator*{\V}{\mathbb{V}}
\DeclareMathOperator*{\LL}{L}
\DeclareMathOperator\erf{erf}
\DeclareMathOperator\trace{tr}
\DeclareMathOperator\median{median}
\DeclareMathOperator*{\argmax}{arg\,max}
\DeclareMathOperator*{\argmin}{arg\,min}
\DeclareMathOperator*{\R}{\mathbb{R}}


\newcommand{\expectation}{\E}
\newcommand{\expectationdata}{\ED}
\newcommand{\variance}{\V}
\let\emptyset\varnothing
\newcommand{\loss}{\LL}
\newcommand{\ascalar}{a}
\newcommand{\bscalar}{b}
\newcommand{\xscalar}{x}
\newcommand{\avec}{{\bf \ascalar}}
\newcommand{\bvec}{{\bf \bscalar}}
\newcommand{\xvec}{{\bf \xscalar}}
\newcommand{\xvectest}{\xvec_{\star}}
\newcommand{\Xmat}{{\bf \MakeUppercase\xscalar}}
\newcommand{\tscalar}{t}
\newcommand{\ttest}{\tscalar_{\star}}
\newcommand{\tvec}{{\bf \tscalar}}
\newcommand{\Tmat}{{\bf \MakeUppercase\tscalar}}
\newcommand{\yscalar}{y}
\newcommand{\yvec}{{\bf \yscalar}}
\newcommand{\Ymat}{{\bf \MakeUppercase\yscalar}}
\newcommand{\wscalar}{w}
\newcommand{\wvec}{{\bf \wscalar}}
\newcommand{\wvecopp}{\wvec^{(-i)}}
\newcommand{\wvecML}{\wvec_{\text{MLE}}}
\newcommand{\wvecMAP}{\wvec_{\text{MAP}}}
\newcommand{\wvecs}{\wvec^{(s)}}
\newcommand{\Wmat}{{\bf \MakeUppercase\wscalar}}
\newcommand{\wbias}{\wscalar_0}
\newcommand{\xn}{\xscalar_{\DataIndex}}
\newcommand{\xvecn}{\xvec_{\DataIndex}}
\newcommand{\tvecn}{\tvec_{\DataIndex}}
\newcommand{\tn}{\tscalar_{\DataIndex}}
\newcommand{\ti}{\tscalar_{i}}
\newcommand{\yvecn}{\yvec_{\DataIndex}}
\newcommand{\yn}{\yscalar_{\DataIndex}}
\newcommand{\yfunc}{\yscalar}
\newcommand{\yfunctest}{\yfunc_{\star}}

\newcommand{\zerovec}{ {\bf 0}}
\newcommand{\pivec}{\boldsymbol{\pi}}
\newcommand{\muvec}{\boldsymbol{\mu}}
\newcommand{\muvecN}{\boldsymbol{\mu}_{\N}}
\newcommand{\munotvec}{\muvec_0}
\newcommand{\mnot}{{\bf m_0}}
\newcommand{\mN}{{\bf m}_{\N}}
\newcommand{\mNplus}{{\bf m}_{\N+1}}
\newcommand{\etavec}{\boldsymbol{\eta}}
\newcommand{\tauvec}{\boldsymbol{\tau}}
\newcommand{\tauinv}{\tau^{-1}}
\newcommand{\thetavec}{\boldsymbol{\theta}}
\newcommand{\thetav}{\thetavec}
\newcommand{\phivec}{\boldsymbol{\phi}}
\newcommand{\Thetamat}{\boldsymbol{\Theta}}
\newcommand{\Gammamat}{\boldsymbol{\Gamma}}
\newcommand{\Gammamatinv}{\Gammamat^{-1}}
\newcommand{\phivectest}{\phivec_{\star}}
\newcommand{\Phimat}{\boldsymbol{\Phi}}
\newcommand{\phivv}{\phivec}
\newcommand{\phivecn}{\phivec_{\DataIndex}}
\newcommand{\phiveci}{\phivec_{i}}
\newcommand{\Sigmamat}{\boldsymbol{\Sigma}}
\newcommand{\SigmamatN}{\Sigmamat_N}
\newcommand{\Sigmamatnot}{\Sigmamat_0}
\newcommand{\Sigmamatinv}{\Sigmamat^{-1}}
\newcommand{\Smat}{{\bf S}}
\newcommand{\SmatN}{\Smat_{\N}}
\newcommand{\Smatnot}{\Smat_0}
\newcommand{\Smatinv}{\Smat^{-1}}
\newcommand{\SmatNplus}{\Smat_{\N+1}}
\newcommand{\SmatNplusinv}{\Smat_{\N+1}^{-1}}
\newcommand{\Betafunc}{\mathcal{B}}
\newcommand{\G}{\mathcal{G}}

\newcommand{\betatest}{\beta_{\star}}
\newcommand{\Amat}{{\bf A}}
\newcommand{\prodnn}{\prod_{n=1}^N}
\newcommand{\prodkk}{\prod_{k=1}^K}
\newcommand{\prodid}{\prod_{i=1}^D}
\newcommand{\proddd}{\prod_{d=1}^D}
\newcommand{\prodiK}{\prod_{i=1}^K}
\newcommand{\sumnn}{\sum_{n=1}^N}
\newcommand{\sumkk}{\sum_{k=1}^K}
\newcommand{\sumw}{\sum_{w}}
\newcommand{\sumww}{\sumw^W}
\newcommand{\sumid}{\sum_{i=1}^D}
\newcommand{\sumiK}{\sum_{i=1}^K}
\newcommand{\znk}{z_{nk}}
\newcommand{\xni}{x_{ni}}
\newcommand{\muki}{\mu_{ki}}
\newcommand{\Bmat}{{\bf B}}
\newcommand{\Cmat}{{\bf C}}
\newcommand{\Dmat}{{\bf D}}
\newcommand{\Emat}{{\bf E}}
\newcommand{\Fmat}{{\bf F}}
\newcommand{\Gmat}{{\bf G}}
\newcommand{\Hmat}{{\bf H}}
\newcommand{\Imat}{{\bf I}}
\newcommand{\Jmat}{{\bf J}}
\newcommand{\Kmat}{{\bf K}}
\newcommand{\Lmat}{{\bf L}}
\newcommand{\Mmat}{{\bf M}}
\newcommand{\Nmat}{{\bf N}}
\newcommand{\Omat}{{\bf O}}
\newcommand{\Pmat}{{\bf P}}
\newcommand{\Qmat}{{\bf Q}}
\newcommand{\Rmat}{{\bf R}}
%\newcommand{\Smat}{{\bf S}}
%\newcommand{\Tmat}{{\bf T}}
\newcommand{\Umat}{{\bf U}}
\newcommand{\Vmat}{{\bf V}}
%\newcommand{\Wmat}{{\bf W}}
%\newcommand{\Xmat}{{\bf X}}
%\newcommand{\Ymat}{{\bf Y}}
\newcommand{\Zmat}{{\bf Z}}

\makeatletter
\newcommand*{\indep}{%
  \mathbin{%
    \mathpalette{\@indep}{}%
  }%
}
\newcommand*{\nindep}{%
  \mathbin{%                   % The final symbol is a binary math operator
    \mathpalette{\@indep}{\not}% \mathpalette helps for the adaptation
                               % of the symbol to the different math styles.
  }%
}
\newcommand*{\@indep}[2]{%
  % #1: math style
  % #2: empty or \not
  \sbox0{$#1\perp\m@th$}%        box 0 contains \perp symbol
  \sbox2{$#1=$}%                 box 2 for the height of =
  \sbox4{$#1\vcenter{}$}%        box 4 for the height of the math axis
  \rlap{\copy0}%                 first \perp
  \dimen@=\dimexpr\ht2-\ht4-.2pt\relax
      % The equals symbol is centered around the math axis.
      % The following equations are used to calculate the
      % right shift of the second \perp:
      % [1] ht(equals) - ht(math_axis) = line_width + 0.5 gap
      % [2] right_shift(second_perp) = line_width + gap
      % The line width is approximated by the default line width of 0.4pt
  \kern\dimen@
  {#2}%
      % {\not} in case of \nindep;
      % the braces convert the relational symbol \not to an ordinary
      % math object without additional horizontal spacing.
  \kern\dimen@
  \copy0 %                       second \perp
} 
\makeatother

\newcommand{\rvec}{{\bf r}}
\newcommand{\betavec}{{\boldsymbol{\beta}}}
\newcommand{\alphavec}{{\boldsymbol{\alpha}}}
\newcommand{\zvec}{{\bf z}}
\newcommand{\zvecn}{{\zvec_n}}
\newcommand{\zvecopp}{{\bf z}^{(-i)}}
\newcommand{\dvec}{{\bf d}}
\newcommand{\lvec}{{\bf l}}
\newcommand{\mvec}{{\bf m}}
\newcommand{\uvec}{{\bf u}}
\newcommand{\vvec}{{\bf v}}

\newcommand{\X}{\mathcal{X}}

\newcommand{\xvecmean}{\bar{\xvec}}
\newcommand{\xvecnest}{\tilde{\xvec}_n}
\newcommand{\xvecestn}{\xvecnest}
\newcommand{\class}{\mathcal{C}}
\newcommand{\gaus}{\mathcal{N}}
\newcommand{\Q}{\mathcal{Q}}
\newcommand{\sigmoid}{\sigma}

\newtheoremstyle{problemstyle}  % <name>
        {1pt}               % <space above>
        {1pt}             % <space below>
        {\normalfont}   % <body font>
        {}              % <indent amount}
        {\bfseries}
        {\normalfont\bfseries:}         % <punctuation after theorem head>
        {.5em}  % <space after theorem head>
        {}  % <theorem head spec (can be left empty, meaning `normal')>
\theoremstyle{problemstyle}{}


\newtheorem{problem}{Problem}
\usepackage{tkz-graph}
\usepackage{tikz}
\usetikzlibrary{bayesnet}
\usepackage{inputenc}
\title{Reinforcement Learning - Homework 1} \date{deadline: November 16, 2018}
\author{Andrii Skliar, 11636785\\ Gabriele Bani, 11636758}

\begin{document}
\maketitle 

% \fbox{
%   \parbox{0.8\textwidth}{
%      During the process of solving the homework problems, I have collaborated with the following colleagues: \\\\
%     % LIST ALL THE COLLABORATORS HERE!
%     \begin{tabular}{c c c c}
%         Gabriele Bani & Gabriele Cesa & Davide Belli & Pascal Esser\\
%         Gautier Dagan & & &
%         % more people? put them here:
%         %  &   &   
%   \end{tabular}\\\\
%   \textit{\small NB: credits for the Latex-format go to Iris Verweij, 2nd year MSc AI Student}.
%   }
% }

% \vspace{0.8cm}

%------------------------------
% Problem 1

\begin{problem}[Exploration]
\ \newline
\begin{enumerate}
    \item In $\epsilon$-greedy action-selection for the case of n actions, what is the probability of selecting the greedy action?
    \begin{solutionorlines}[2in]
        Probability of selecting the greedy action is $1 - \epsilon$.
    \end{solutionorlines}
    \item Consider a 3-armed bandit problem with actions 1, 2, 3. If we use $\epsilon$-greedy action-selection, initialization at 0, and sample-average action-value estimates, which of the following sequence of actions are certain to be the result of exploration? $A_1 = 1, R_1 = −1, A_2 = 2, R_2 = 1, A_3 = 2, R_3 = −2, A_4 = 2, R_4 = 2, A_5 = 3, R_5 = 1$.
    \begin{solutionorlines}[2in]
        Actions $A_4, A_5$ are results of exploration. This can be seen by calculating average action-value estimates at each timestep and seeing when the action taken is not the one that yields best estimates.
    \end{solutionorlines}
    \item You are trying to find the optimal policy for a two-armed bandit. You try two approaches: in the pessimistic approach, you initialize all action-values at $-5$, and in the optimistic approach you initialize all action-values at $+5$. One arm gives a reward of $+1$, one arm gives a reward of $-1$. Using a greedy policy to choose actions, compute the resulting Q-values for both actions after three interactions with the environment. In case of a tie between two Q-values, break the tie at random.
    \begin{solutionorlines}[2in]
    We assume that arm $A_1$ gives reward of $+1$ and arm $A_2$ gives reward of $-1$.\\
    \begin{center}
    \def\arraystretch{1.5}
    \begin{tabular}{|c|c|c|l|}
    \hline
    Step & $Q_1 (+1)$ & $Q_2 (-1)$ & Action Taken \\ \hline
    0 & $-5$ & $-5$ & \begin{tabular}[c]{@{}l@{}}All action values are the same, select arm randomly. \\ Assume, $Q_1$ has been selected.\end{tabular} \\ \hline
    1 & $1$ & $-5$ & Greedily choose $Q_1$ \\ \hline
    2 & $1$ & $-5$ & Greedily choose $Q_1$ \\ \hline
    3 & $1$ & $-5$ & \multicolumn{1}{c|}{} \\ \hline
    \end{tabular}
    \end{center}
    \newpage
    
    \begin{center}
    \def\arraystretch{1.5}
    \begin{tabular}{|c|c|c|l|}
    \hline
    Step & $Q_1 (+1)$ & $Q_2 (-1)$ & Action Taken \\ \hline
    0 & $5$ & $5$ & \begin{tabular}[c]{@{}l@{}}All action values are the same, select arm randomly. \\ Assume, $Q_1$ has been selected.\end{tabular} \\ \hline
    1 & $1$ & $5$ & Greedily choose $Q_2$ \\ \hline
    2 & $1$ & $-1$ & Greedily choose $Q_1$ \\ \hline
    3 & $1$ & $-1$ & \multicolumn{1}{c|}{} \\ \hline
    \end{tabular}
    \end{center}
    
    \end{solutionorlines}
    
    \begin{centering}
    \begin{table}[]
    
    \end{table}
    \end{centering}
    
    \item Which initialization leads to a higher (undiscounted) return? What if you had broken the tie differently?
    \begin{solutionorlines}[2in]
        Pessimistic initialization leads to higher return because in this case it will lead to pure exploitation without any exploration. However, if we would break tie choosing $Q_2$ in the first step, we would never choose $Q_1$ thus getting only negative rewards. 
        Optimistic initialization, on the other hand, will lead to the same return even with any break in the first step (at least for the first 3 steps). Asymptotically, its performance should also be better due to the fact described in the next part of the question. 
    \end{solutionorlines}
    \item Which initialization leads to a better estimation of the Q-values?
    \begin{solutionorlines}[2in]
        Optimistic initialization leads to a better estimation of the Q-values as, unlike pessimistic initialization, it allows for exploration and not just pure exploitation strategy. Pessimistic initialization asymptotically can only lead to a good estimation of a Q-value of a single hand (the one chosen first), while optimistic initialization can asymptotically lead to a good estimation of all Q-values.
    \end{solutionorlines}
    
    \item Explain why one of the two initialization methods is better for exploration.
    \begin{solutionorlines}[2in]
        Optimistic initialization is better for exploration due to the fact that it makes initial Q-value larger than any possible action-value, so after taking any action, greedy policy forces us to use other, not yet explored actions as their action-value is higher. At the same time, pessimistic initialization will choose the same action at all timesteps. It is due to the fact that all the initial action-values are smaller than then action-value of an action chosen at the first timestep and will enforce that only this action will be chosen at any future timestep.
    \end{solutionorlines}
\end{enumerate}

\end{problem}

%--------------------------------------

\bibliographystyle{plain}
\bibliography{bibliography}
\end{document}